\documentclass[margin,line]{CV}

\usepackage{graphicx}
\usepackage{wrapfig}
\usepackage{ifthen}
\usepackage{hyperref}

\newboolean{foreign}
\setboolean{foreign}{false}

%\def\superofficial{}
\def\consentclause{}

\begin{document}
\name{\Large Alexander Fokin}
\begin{resume}

%\begin{wrapfigure}{r}{2cm}
%    \vspace{-20pt}
%    \includegraphics[width=2cm]{photo.jpg}
%    \vspace{-20pt}
%\end{wrapfigure}

    \section{\mysidestyle Contact\\Information}
    Current Location: Moscow, Russian Federation \\
    Mobile: \includegraphics[height=0.35cm]{phone-ru.png} \\ 
    E-mail: apfokin@gmail.com \\
    LinkedIn: \url{https://linkedin.com/in/retgone}

    \section{\mysidestyle Professional\\Experience}
    \textbf{ISO C++ Standards Committee} \vspace{2mm}\\\vspace{1mm}%
    \textsl{C++ Expert} \hfill \textbf{May 2016 - present}\\\vspace{1mm}%
    Currently serving as a chairman of the C++ standards working group of the Russian National Body.

\ifdefined\superofficial
    {\footnotesize\textit{ISO C++ Standards Committee is called WG21, officially ISO/IEC JTC1 (Joint Technical Committee 1) / SC22 (Subcommittee 22) / WG21 (Working Group 21). More info is available at \url{http://isocpp.org} and \url{http://stdcpp.ru}.}}
\fi
    
    \textbf{Yandex}, Moscow, Russian Federation \vspace{2mm}\\\vspace{1mm}%
    \textsl{Head of Search Components} \hfill \textbf{October 2015 - present}\\\vspace{1mm}%
    \textsl{Senior Search Engineer} \hfill \textbf{October 2014 - October 2015}\\
    Currently responsible for most of Yandex web search runtime, from L7 load balancer all the way down to the services running on top of search database shards, managing a team of some 80 engineers.
    
    Here is a list of some of the most important projects that I have worked on:
    
    \begin{itemize}
    \item Increasing search performance and number of searchable documents. This is the project that I was hired into as an engineer, and it started as a rewrite of one of the search stages. I wrote most of the foundational code and shipped the first version, bringing significant performance gains. Throughout the next several years we continued to ship incremental improvements, considerably increasing the number of document in the search database while simultaneously bringing substantial performance gains.
    \item Improving technical interviews in the company, making them standardized and consistent. This is a company-wide project that has included everything from researching how it's done in other companies, to doing lectures on best interviewing practices, preparing 100s of interviewers, doing 100s of interviews myself, running a Q\&A channel, creating an online seminar, incrementally improving the rules, and scaling it all up when I was no longer able to do everything by myself. 
    \item Revitalizing the codebase and creating a system that would motivate developers to pay off technical debt in common code. This is also a company-wide project that started with me porting clang's libcxx to MSVC so that we could switch to C++11 STL, and realizing that this effort could be crowdsourced. This then paved way to some important repository-wide refactorings that could not have been possible otherwise.
    \item Refactoring the search stack and replacing all the different ways services were communicating with each other with a single unified graph-based approach.
    \item Switching from an old system with separate admins \& developers into a more efficient one where developers are the ones maintaining their services.
    \item Improving our infrastructure to make it possible to considerably increase the release frequency of key search components, and provide SLAs for the time it takes for a commit to get into production.
    \end{itemize}

\ifdefined\superofficial
    {\footnotesize\textit{Yandex is one of the largest internet companies in Europe, operating Russia's most popular search engine. More info is available at \url{https://yandex.com/company/}.}}
\fi 

\ifdefined\superofficial
    \pagebreak
\fi
\ifdefined\superofficial\else
    \pagebreak    
\fi
    
    \textbf{Network Optix}, Moscow, Russian Federation, then Los Angeles, USA \vspace{2mm}\\\vspace{1mm}%
    \textsl{Senior Software Engineer} \hfill \textbf{October 2011 - July 2014}\\ 
    Designed and implemented initial version of the HD Witness client application, making sure that its high-level architecture is sound and extensible. As of 2016, five years down the road, most of the foundational code is still unchanged, with a bunch of features added on top.

    As the person solely responsible for the client-facing part of the system, I made no compromises when it came to delivering the best experience for our users. After 1.0 release various sources have described HD Witness as the most user-friendly and aesthetically pleasing video management system on the market, which has helped the company to gain a competitive edge.

    Was subsequently charged with management of the front-end development team. Other responsibilities included design of public APIs and development of generic C++ libraries that were used internally.
    
    %Streamlined the UI making it highly intuitive and utilized OpenGL to provide a fluid and visually appealing user experience. 
    
    %Was subsequently responsible for development of generic C++ libraries that were used internally, including compile-time reflection for C++ types, data serialization and object-relational mapping. Other responsibilities included design of public APIs, management of the front-end development team and ensuring UI and UX consistency of our desktop and mobile clients.

\ifdefined\superofficial
    {\footnotesize\textit{Network Optix is an enterprise video software development company headquartered in Burbank, CA, focused on building easy-to-use video management technologies. More info is available at \url{http://www.networkoptix.com/}.}}
\fi
    
    %\textbf{Combild}, Moscow, Russian Federation \vspace{2mm}\\\vspace{1mm}%
    %\textsl{Software Development Lead, Co-founder} \hfill \textbf{June 2010 - October 2011}\\
    %Combild was started to create an IT service management (ITSM) system that would target small companies and IT outsourcers, an underserved market niche for which competing solutions were either too expensive or excessively complex. Combild was to offer a lightweight and user-friendly ITSM system with a licensing model specifically targeted at small companies and IT oursourcers.

    %At the time all other solutions on the market were either too bulky and hard to maintain, or were not suitable for the business model of IT outsourcers. Frustrated with the state of things, we have decided to roll out our own product.

    %My role was to lay out the initial product architecture, to implement the first demonstrable version, and then to work with customers to prioritize and clarify the features and to manage a small development team. 
    %The system was developed in C++/Qt and targeted both Windows and Linux.
    
    
    \textbf{SmartDec}, Moscow, Russian Federation \vspace{2mm}\\\vspace{1mm}%
    \textsl{Software Engineer} \hfill \textbf{July 2009 - September 2011}\\
    Was mainly working on SmartDec, a native code decompiler. Laid out the architecture of the decompiler and implemented several frontend and backend plugins, including support for different x86 and PIC assembly input formats. Was responsible for devising novel algorithms that would improve the quality of the decompiled code and would allow for reconstruction of C++-specific constructs. This effort has led to several publications on international conferences on reverse engineering.

    Have also implemented a form recognition toolkit that was subsequently used in some of the Moscow schools for test checking. 
    %Detailed description is available at \url{http://elric.ru/wordpress/projects/form-recognition-toolkit/}. 

    Was additionally working on \url{http://mathege.ru}, a national mathematics exam portal. Did both frontend and backend development and have implemented a \LaTeX~to html converter that was used for importing problems into the system.

\ifdefined\superofficial
    {\footnotesize\textit{SmartDec is a software security consulting company specializing in vulnerability analysis and reverse engineering. More info is available at \url{http://smartdec.net}.}}
\fi
    
    
    \textbf{Institute for System Programming of the Russian Academy of Sciences}, Moscow, Russian Federation \vspace{2mm}\\\vspace{1mm}%
    \textsl{Software Engineer} \hfill \textbf{September 2007 - September 2008}\\
    Was working in a team developing a framework for dynamic analysis of binary code. Using C++ metaprogramming techniques implemented a disassembler for MIPS64 architecture that significantly outperformed all other disassemblers for this architecture.

\ifdefined\superofficial
    {\footnotesize\textit{ISPRAS is an R\&D organization in the field of system programming and software engineering. Among its clients are leading IT companies such as HP, IBM, Intel and others. More info is available at \url{http://www.ispras.ru/en/}.}}
\fi
    
   
    \textbf{Intel}, Moscow, Russian Federation \vspace{2mm}\\\vspace{1mm}%
    \textsl{Software Engineering Intern} \hfill \textbf{February 2007 - April 2008}\\
    Was researching computer vision algorithms and have implemented a panorama stitching application. Description is available at \url{https://github.com/retgone/prec}.

    Was also charged with the development of Ruby bindings for Intel's Integrated Performance Primitives library. Description is available at \url{https://github.com/retgone/ipp4r}.
    
\ifdefined\superofficial
    {\footnotesize\textit{Intel is an American multinational corporation and technology company. More info is available at \url{http://www.intel.com}.}}
\fi
    
    
    %\textbf{Personal Projects} \vspace{2mm}\\\vspace{1mm}%
    %I am an avid programmer and I enjoy writing code in my free time. Throughout the years I have done a lot freelance work and have finished several personal projects, including a real-time ray-tracing engine, a virtual mouse driver for Windows XP, a tool for automatic reconstruction of 3d solids from engineering drawings and a lot of OpenGL demos. For more information check out my github page (\url{https://github.com/retgone}).
   
   
    \pagebreak    

   
    \section{\mysidestyle Education}
    \textbf{Department of Computational Mathematics and Cybernetics, Moscow State University}, Moscow, Russian Federation \vspace{2mm}\\\vspace{1mm}%
    \textsl{\ifthenelse{\boolean{foreign}}{Bachelor's}{Specialist} degree in Applied Mathematics and Computer Science} \hfill \textbf{September 2004 - July 2009}\vspace{1mm}\\
    Advisor: Professor Alexander Chernov \\
    Thesis: Reconstruction of Class Hierarchies for Decompilation of C++ Programs \\
    Graduated with high honors. Diploma GPA is 5.0 out of 5.0.

    \textbf{Graduate School of Science and Engineering, Chuo University}, Tokyo, Japan \vspace{2mm}\\\vspace{1mm}%
    \textsl{Full-time non-degree student} \hfill \textbf{September 2008 - March 2009}\vspace{1mm}\\
    Advisor: Professor Mitsunori Makino \\
    Was studying Japanese, working on algorithms for real-time ray tracing and implemented a real-time ray tracer for use with CAVE automatic virtual environment.

    
%    \section{\mysidestyle Research\\Interests}
%    Image-based modeling and rendering, 3d reconstruction.
%    Ray tracing and global illumination, especially in real time.
%    Software reverse engineering, binary translation, decompilation.

%    \section{\mysidestyle Research\\Experience}
%    \textbf{SmartDec} \vspace{2mm}\\\vspace{1mm}%
%    \hfill \textbf{October 2010 - September 2011}\\
%    I continued my work on C++ decompilation at SmartDec. 

%    \textbf{Institute for System Programming of the Russian Academy of Sciences} \vspace{2mm}\\\vspace{1mm}%
%    \hfill \textbf{July 2008 - September 2010}\\
%    I was doing research on decompilation of C++ programs.


    \section{\mysidestyle Publications}
    A. Fokin, E. Derevenetc, A. Chernov and K. Troshina. ``SmartDec: Approaching C++ Decompilation'',
    in proceedings of the \textsl{18th Working Conference on Reverse Engineering}, pp. 347-356, 2011.

    A. Fokin, K. Troshina and A. Chernov. ``Reconstruction of Class Hierarchies for Decompilation of C++ Programs'',
    in proceedings of the \textsl{14th European Conference on Software Maintenance and Reengineering}, pp. 249-252, 2010.

    K. Troshina, A. Chernov and A. Fokin. ``Profile-Based Type Reconstruction for Decompilation'',
    in proceedings of the \textsl{17th International Conference on Program Comprehension}, pp. 263-267, 2009.


    \section{\mysidestyle Honours, \\Awards and \\Test Scores}
    TOEFL iBT, 111/120, Moscow, 2010.                                                               \vspace{1mm}\\
    M.V. Lomonosov Scholarship for Academic Excellence, Moscow, 2006-2009.                          \vspace{1mm}\\
    ABBYY Collegiate Mathematics Competition, 1st place, Moscow, 2006.                              \vspace{1mm}\\
    8th Moscow Collegiate Programming Contest, 9th place, Moscow, 2006.                             \vspace{1mm}\\
    7th Moscow Collegiate Programming Contest, 11th place, Moscow, 2005.                            \vspace{1mm}\\
    Unified State Exam in Mathematics, 100/100 (nationwide top), Izhevsk, 2004.                     \vspace{1mm}

%    \section{\mysidestyle Professional\\Skills}
%    Programming Languages: substantial experience with C++ and Java; experience with Delphi, x86 Assembly; some experience with C\#, Matlab, Perl and Python. \\
%    Libraries and Framewords: considerable experience with modern C++ libraries and frameworks such as Qt and boost. \\
%    Platforms: Windows, Linux, Mac OS X. A lot of experience writing cross-platform code. \\
%    Databases: MySQL, SQLite.

    \section{\mysidestyle Languages}
    Russian: native. \\
    English: fluent. \\
    Japanese: intermediate.
  
    \ifdefined\consentclause
    \vspace{5mm}
    {\footnotesize\textit{I hereby give consent for my personal data included in my application to be processed for the purposes of the recruitment process under the Personal Data Protection Act as of 29 August 1997, consolidated text: Journal of Laws 2016, item 922 as amended.}}
    \fi
    
    
\end{resume}
\end{document}



















